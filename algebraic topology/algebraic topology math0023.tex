\documentclass[a4paper,14pt]{extarticle}
\usepackage{color}
\usepackage{amsmath}
\usepackage{amssymb}
\usepackage{centernot}

\begin{document}


\title{\textbf{Algebraic Topology - MATH0023}}
\author{\textbf{Based on lectures by Prof FEA Johnson}\\ Notes taken by Imran Radzi}
\date{}
\maketitle

\pagenumbering{roman}
Notes based on the Autumn 2021 Algebraic Topology lectures by Prof FEA Johnson.
\begingroup
\let\cleardoublepage\clearpage
\tableofcontents
\endgroup
\newpage
\pagenumbering{arabic}

\vspace{12pt}

\section{Simplicial complexes}

\noindent\textbf{Definition.} A \textit{simplicial complex} $X$ is a pair $(V_X,\mathcal{S}_X)$ where $V_X$ denotes the vertex set of $X$ and $\mathcal{S}_X$ is the set of
\textit{finite, non-empty} subsetse of $V_X$ satisfying
\begin{enumerate}
	\item $\forall v\in V_X$, then $\{v\}\in\mathcal{S}_X$
	\item If $\sigma\in\mathcal{S}_X, \,\tau\subset\sigma, \,\tau\neq\emptyset$, then $\tau\in\mathcal{S}_X$. 
\end{enumerate}
$\mathcal{S}_X$ is called the set of \textit{simplices} of $X$. \\

\vspace{12pt}

\noindent\textbf{Examples.} \\

\noindent A standard 1-simplex, denoted by $\Delta^1$ is simply the line segment (or usually denoted by $I$). 
\[V_{\Delta^1}=\{0,1\}\] \[\mathcal{S}_{\Delta^1}=\{\{0\},\{1\},\{0,1\}\}\]

\vspace{12pt}

\noindent A standard 2-simplex, denoted by $\Delta^2$ is the equilateral triangle.
\[V_{\Delta^2}=\{0,1,2\}\] \[\mathcal{S}_{\Delta^2}=\{\{0\},\{1\},\{2\},\{0,1\},\{0,2\},\{1,2\},\{0,1,2\}\}\]

\vspace{12pt}

In general, the \textit{standard $n$-simplex} $\Delta^n$, is $\Delta^n=(V_{\Delta^n},\mathcal{S}_{\Delta^n})$ where
\[V_{\Delta^n}=\{0,1,\ldots,n\}\] \[\mathcal{S}_{\Delta^n}=\{\alpha:\alpha\subset\{0,\ldots,n\}, \,\alpha\neq\emptyset\}\]




\end{document}