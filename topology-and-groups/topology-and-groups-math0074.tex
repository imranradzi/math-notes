\documentclass[a4paper,14pt]{extarticle}
\usepackage{color}
\usepackage{amsmath}
\usepackage{amsthm}
\usepackage{amssymb}
\usepackage{bbold}
\usepackage{mathtools}
\usepackage{centernot}
\usepackage{tikz}
\usepackage{tikz-cd}
\usepackage{caption}
\usepackage{adjustbox}
\usepackage{multicol}  

\theoremstyle{definition}
\newtheorem*{theorem}{Theorem}
\newtheorem*{definition}{Definition}
\newtheorem*{lemma}{Lemma}
\newtheorem*{corollary}{Corollary}
\newtheorem*{proposition}{Proposition}
\newtheorem*{eg}{Example}
\newtheorem*{remark}{Remark}


\begin{document}

\title{\textbf{Topology and Groups - MATH0074}}
\author{\textbf{Based on lectures by Dr. Lars Louder}\\ Notes taken by Imran Radzi}
\date{}
\maketitle

\pagenumbering{roman}
(Revision) notes based on the Autumn 2021 Topology and Groups lectures by Dr. Lars Louder.
Some parts marked with (*) (+) are taken from Hatcher's Algebraic Topology and 
the site on Topology and Groups by Prof. Jonny Evans.
\begingroup
\let\cleardoublepage\clearpage
\tableofcontents
\endgroup
\newpage
\pagenumbering{arabic}

\section{Point-set Topology}

\subsection{Preliminaries}
\begin{definition}[Topological space]
  A topological space is a pair $(X,\mathcal{T})$ such that
  \begin{enumerate}
    \item $X$ is a set 
    \item $\mathcal{T}\subset \mathcal{P}(X)$ is a collection of subsets of $X$
    \item $\emptyset\in\mathcal{T}, \,X\in\mathcal{T}$
    \item $\mathcal{T}$ is closed under finite intersections and arbitrary unions
  \end{enumerate}
\end{definition}

\begin{definition}[Open neighbourhood]
  If $x\in X, \,U$ open in $X$, and $x\in U$, then $U$ is an \emph{open neighbourhood} of $x$.
\end{definition}

\begin{definition}[Hausdorff spaces]
  A topological space $(X,\mathcal{T})$ is \emph{Hausdorff} if $\forall \,x,y\in X$, there 
  exists $U,V$ open neighbourhoods of $x,y$ respectively such that $U\cap V=\emptyset$.
\end{definition}

\begin{definition}[Homeomorphisms]
  A map $f:X\rightarrow Y$ is a \emph{homeomorphism} if 
  \begin{enumerate}
    \item $f$ is bijective
    \item $f$ is continuous
    \item $f^{-1}$ is continuous
  \end{enumerate}
\end{definition}

\begin{definition}[Continuous maps]
  A map $f:X\rightarrow Y$ is continuous if $\forall \,U\text{ (open)}\subset Y$,
  $f^{-1}(U)$ is open in $X$.
\end{definition}

\begin{definition}
  If $\mathcal{T}$ and $\mathcal{T}'$ are topologies on $X$ such that $\mathcal{T}\subsetneq\mathcal{T}'$ then $\mathcal{T}'$ is \emph{finer} than $\mathcal{T}$, and 
  $\mathcal{T}$ is \emph{coarser} than $\mathcal{T}'$.
\end{definition}

\begin{proposition}
  $\text{id}:(X,\mathcal{T}\rightarrow (X,\mathcal{T}')$ is continuous if and only if 
  $\mathcal{T}$ is finer than $\mathcal{T}'$.
\end{proposition}

\begin{definition}[Subspace topology]
  If $X$ is a topological space, $Y\subset X$, the subspace topology on $Y$ is defined by
  \[U\text{ open in } Y\iff \exists V\text{ open in }X \text{ such that }U=Y\cap V\]
\end{definition}

\begin{definition}
  If a map $f:X\rightarrow Y$ is continuous, the \emph{image} of $f$ is the set 
  \[f(X)=\{f(x)\,|\,x\in X\}\subset Y\] with the subspace topology.
\end{definition}

\begin{definition}[Product topology]
  Let $X, Y$ be spaces. The \emph{product topology} on $X\times Y$ is the smallest (coarsest)
  topology making the projections
  \[p_X:X\times Y\rightarrow X, \,\,\,p_Y:X\times Y\rightarrow Y\]
  continuous.
\end{definition}

\begin{proposition}
  Product of Hausdorff spaces if Hausdorff.
\end{proposition}

\subsection{Connectedness}
\begin{definition}[Connectedness]
  A space $X$ is \emph{disconnected} if there exists a surjective continuous map 
  $f:X\rightarrow\{p_1,p_2\}$. A space is \emph{connected} if every continuous function
  $f:X\rightarrow\{p_1,p_2\}$ is constant.
\end{definition}

\begin{definition}
  A pair of sets $U,V\subset X$ is said to disconnect $X$ if they are non-empty, disjoint,
  $U\cup V=X$ and both are open.
\end{definition}

\begin{definition}
  $X$ is disconnected if there exists $U,V$ which disconnect $X$.
\end{definition}


\begin{definition}[Path]
  A \emph{path} in $X$ is a continuous map $\gamma:[0,1]\rightarrow X$. $\gamma$ is a path 
  from $\gamma(0)$ to $\gamma(1)$. $a,b\in X$ are said to be connected by a path if there 
  is a path from $a$ to $b$.
\end{definition}

\begin{definition}[Path-connectedness]
  A space $X$ is \emph{path-connected} if for all $x,y$, there exists
  \[\gamma:[0,1]\rightarrow X\text{ such that }\gamma(0)=x, \,\gamma(1)=y\]
\end{definition}

\noindent or equivalently,

\begin{definition}
  We say $X$ is path-connected if there exists a unique equivalence class, where the 
  equivalence relation $\sim$ is defined $a\sim b$ if and only if there exists a path from 
  $a$ to $b$.
\end{definition}

\begin{proposition}
  Suppose $X$ is connected. Then, if $f:X\rightarrow Y$, then $f(X)\subset Y$ is connected.
\end{proposition}

\begin{proposition}
  $[0,1]$ is connected.
\end{proposition}

\begin{corollary}
  If $X$ is path-connected, then $X$ is connected.
\end{corollary}

\begin{definition} $X\subset\mathbb{R}$ is an \emph{interval} if $a\leq b\leq c$, 
  $a,c\in X\implies b\in X$. 
\end{definition}

\begin{proposition}
  A subset of $\mathbb{R}$ is connected if and only if it is an interval.
\end{proposition}

\begin{definition}[Locally (path) connected]
  A space $X$ is locally (path ) connected at a point $p$ if for every open neighbourhood $U$ of $p$, 
  there exists a (path) connected open neighbourhood $V$ of $p$ such that $p\in V\subset U$.
\end{definition}

\begin{proposition}
  If $X$ is locally path-connected then the path components of $X$ are open.  
\end{proposition}

\begin{proposition}
  If $X$ is connected and locally path-connected, then $X$ is path connected.
\end{proposition}


\subsection{Compactness}
\begin{definition}[Open cover]
  An \emph{open cover} of a space $X$ is a collection of open sets $\mathcal{U}$ such that
  \[X=\bigcup_{U\in\mathcal{U}}U\]
\end{definition}

\begin{definition}
  A space $X$ is \emph{compact} if every open cover has a finite subcover.
\end{definition}

\begin{lemma}
  Closed subset sof compact spaces are compact.
\end{lemma}

\begin{theorem}
  If $X,Y$ are compact, then $X\times Y$ is compact.
\end{theorem}

\begin{theorem}[Heine-Borel theorem]
  $X\subset\mathbb{R}^n$ is compact if and only if $X$ is closed and bounded.
\end{theorem}

\begin{theorem}
  $[0,1]$ is compact.
\end{theorem}

\begin{theorem}
  If $f:X\rightarrow Y$ is continuous, $X$ compact, then $f(X)\subset Y$ is compact with respect to the subspace topology.
\end{theorem}

\begin{proposition}
  If $C\subset Y$ is compact, $Y$ Hausdorff, then $C$ is closed.
\end{proposition}

\begin{proposition}
  If $f:X\rightarrow Y$ is a continuous bijection, $X$ compact, $Y$ Hausdorff, then $f$ 
  is a homeomorphism
\end{proposition}

\subsection{Quotient spaces}
\begin{definition}[Quotient map]
  Let $q:X\rightarrow Y$ be a continuous surjection. Then $q$ is a \emph{quotient map}
  if $q^{-1}(Y)$ is open if and only if $U$ is open. (A bijective quotient map is a 
  homeomorphism)
\end{definition}

\begin{definition}[Quotient space]
  Let $X$ be a space, and $\sim$ an equivalence relation on $X$, and 
  $q:X\rightarrow X/\sim = Y$ the quotient map. The quotient topology on $Y$ is defined by
  $U$ open in $Y$ if and only if $q^{-1}(U)$ is open in $X$.
\end{definition}

\begin{lemma}
  \[
  \begin{tikzcd}
    X \arrow{r}{f} \arrow[swap]{dr}{q} & Z  \\
     & Y \arrow{u}{h}
  \end{tikzcd}
\]
  Let $f$ be continuous, and suppose $f$ factors through $q:X\rightarrow Y$, a quotient 
  map, i.e., $\exists h:Y\rightarrow Z$ such that $h\circ q = f$. Then $h$ is continuous.
\end{lemma}

\begin{proposition}
  Let $f:X\rightarrow Y$ be a continuous surjection with $X$ compact, $Y$ Hausdorff. Then 
  $f$ is a quotient map. 
\end{proposition}

\begin{definition}[Disjoint union]
  Let $X_1,X_2$ be topological spaces. The \emph{disjoint union} of $X_1$ and $X_2$,
  $X_1\sqcup X_2$ is the space with the underlying set $X_1\sqcup X_2$, with $U$ open 
  in $X_1\sqcup X_2$ if and only if $U\cap X_1$ is open in $X_1$, and $U\cap X_2$ is 
  open in $X_2$.
\end{definition}

\begin{definition}[Cell complex]
  A \emph{cell complex} is a space built up inductively, as follows
  \begin{enumerate}
    \item ($n=0$) We start with a discrete set $X^{(0)}$ consisting of points, which we call 
    $0$-cells $\{e_i^0\,|\,i\in I_0\}, \,e_i^0\cong pt$. $X^{(0)}=\bigsqcup_i e_i^0$ is called the $0$-skeleton.
    \item ($n>0$) We add a (possibly empty) subset of $n$-cells $\{e_i^n\,|\,i\in I_n\}$ 
    $e_i^n\cong D^n$, the $n$-dimensional disk, and a continuous map 
    \[\phi_i^n:\partial e_i^n\cong S^{n-1}\rightarrow X^{(n-1)}\] and here the $n$-skeleton 
    is \[X^{(n)}=X^{(n-1)}\sqcup\bigsqcup e_i^n/\sim\]
  \end{enumerate}
  A space $X$ is a cell complex if there exists $X^{(0)}\subset X^{(1)}\subset\ldots$ 
  as above, with the condition that $U$ is open in $X$ if and only if 
  $X^{(n)}\cap U$ is open for all $n$. \\

  $X^{(0)}\subseteq X^{(1)}\subseteq\ldots$ is called the \emph{cell decomposition} of $X$.
\end{definition}

\begin{definition}
  The suspension $SX$ of a space $X$ is the space
  \[SX=X\times I/\sim\] where $(x,t)\sim (x',t')$ if and only if $(x,t)=(x',t')$ or 
  $t=t'=1$ or $t=t'=0$.
\end{definition}

\begin{proposition}
  $SS^n$ is homeomorphic to $S^{n+1}$. $SD^n$ is homeomorphic to $D^{n+1}$.
\end{proposition}

\begin{definition}[Presentation complex]
  text
\end{definition}

\begin{definition}[Cayley graph]
  text
\end{definition}

\section{Homotopy}

\subsection{Homotopy}

\begin{definition}
  Let $(X,A)$ be a pair of spaces, where $A\subseteq X$, $f_0,f_1:X\rightarrow Y$. We say
  $f_0$ and $f_1$ are \emph{homotopic relative} to $A$ if there exists a function $F$ 
  $F:X\times I\rightarrow Y$ such that $F(-,0)=f_0, \,F(-,1)=f_1$ and $F(a,t)=f_0(a)=f_1(a)$ 
  for all $t$. In this case we write $f_0\simeq_A f_1$. \\

  If $A=\emptyset$ then we say $f_0$ and $f_1$ are \emph{homotopic} and write $f_0\simeq f_1$.
\end{definition}

\begin{lemma}[*]
  A function defined on the union of two closed sets is continuous if it is continuous when 
  restricted to each of the closed sets separately.
\end{lemma}

\begin{proposition}
  Any two continuous maps $f_0,f_1:X\rightarrow \mathbb{R}^n$ are homotopic via the homotopy
  \[F(x,t)=tf_1(x)+(1-t)f_0(x)\]
\end{proposition}

\begin{definition}[Homotopy equivalence]
  Two spaces $X$ and $Y$ are \emph{homotopy equivalent} if there exists 
  $f:X\rightarrow Y$, $g:Y\rightarrow X$ such that 
  $f\circ g\simeq\text{id}_Y$, $g\circ f\simeq\text{id}_X$. In this case, we write 
  $X\simeq Y$.
\end{definition}

\begin{proposition}
  Homotopy equivalence is an equivalence relation on (topological) spaces.
\end{proposition}

\begin{proposition}
  $\mathbb{R}^n\simeq pt$
\end{proposition}

\begin{definition}
  A space $X$ is \emph{contractible} if $X\simeq pt$, or in other words,
  $\text{id}:X\rightarrow X$ is homotopic to 
  a constant map. In this case the map $\text{id}_X$ is said to be \emph{null-homotopic}.
\end{definition}

\begin{proposition}
  $\mathbb{R}^n\setminus pt\simeq S^{n-1}$
\end{proposition}

\begin{proposition}
  If $f:X\rightarrow S^2$ is a non-surjective map then $f$ is homotopic to a constant 
  map.
\end{proposition}

\begin{definition}
  The cone $CX$ on a space $X$ is the space
  \[CX=X\times I/\sim\] where $(x,t)\sim(x',t')$ if and only if $(x,t)=(x',t')$ or 
  $t=t'=1$.
\end{definition}

\begin{proposition}
  $CX$ is always contractible.
\end{proposition}

\begin{proposition}
  If $X$ is contractible then $X$ is path-connected.
\end{proposition}

\begin{definition}[Retract]
  Let $A\subseteq X$ be a subspace. $A$ is a \emph{retract} of $X$ if there exists a continuous map $f:X\rightarrow A$ (retraction) such that $r|_A=\text{id}_A$. $A$ is a \emph{deformation
  retract} of $X$ if there exists such a function $r$ such that $r$ is homotopic to 
  $\text{id}_X$ relative to $A$.
\end{definition}

\begin{proposition}
  If $A$ is a deformation retract of $X$ then $X\simeq A$.
\end{proposition}

\subsection{Paths and path-homotopy}

\begin{definition}[Path-homotopy]
  Two paths $\gamma_0$ and $\gamma_1$ are \emph{path-homotopic} if they are homotopic 
  relative to $\{0,1\}\subseteq I$. In particular $\gamma_0(0)=\gamma_1(0)$, 
  $\gamma_0(1)=\gamma_1(1)$. If $F$ is a homotopy from $\gamma_0$ to $\gamma_1$,
  \[F(-,0)=\gamma_0(0), \,F(-,1)=\gamma_1(1)\] $F$ is a family of paths connecting 
  $\gamma_0(0)$ and $\gamma_0(1)$
\end{definition}

\begin{proposition}
  Path-homotopy is an equivalence relation on the set of paths in (a topological space) $X$.
\end{proposition}

\begin{definition}[Based loop]
  A \emph{based loop} at $x_0\in X$ is a path $\gamma:I\rightarrow X$ such that 
  $\gamma(0)=\gamma(1)=x_0$.
\end{definition}

\begin{definition}[Fundamental group of a space]
  The \emph{fundamental group} of $X$ at $x_0$ is the set (group)
  \[\{[\gamma]\,|\,\gamma\text{ is a loop based at }x_0\}\]
  which is denoted by $\Pi_1(X,x_0)$.
\end{definition}

\begin{definition}[$n^{\text{th}}$ homotopy group]
  The $n^{\text{th}}$ homotopy group of a space $X$ at $x_0$ is the set (group)
  \[\pi_n(X,x_0)=\{[f:I^n\rightarrow X\,|\,f(\partial I^n)\rightarrow x_0]\}\]
\end{definition}

\begin{definition}
  A loop based at $x_0$ is null-homotopic if it is path-homotopic to a constant path.
\end{definition}

\begin{definition}[Free homotopy]
  If $\gamma_0$ and $\gamma_1$ are based loops (not necessarily at the same point), then 
  $\gamma_0$ and $\gamma_1$ are \emph{freely homotopic} if they are homotopic through 
  based loops, so if $F$ is a free homotopy between $\gamma_0$ and $\gamma_1$, then,
  \[F(x_0)=\gamma_0, \,F(x,1)=\gamma_1\] \[F(0,t)=F(1,t)\text{ for all }t\]
\end{definition}

\begin{proposition}
  Free homotopy is an equivalence relation on the set of based loops in (a topological space)
  $X$.
\end{proposition}

\begin{definition}
  A based loop \emph{bounds a disk} if the induced map \[\bar{\gamma}:[0,1]/
  {\scriptstyle 0=1}\cong S^1
  \subseteq D^2\] extends to a continuous function $D^2\rightarrow X$.
\end{definition}

\begin{lemma}
  The following are equivalent
  \begin{enumerate}
    \item $\gamma$ bounds a disk. 
    \item $\gamma$ is null-homotopic.
    \item $\gamma$ is freely homotopic to a constant path.
  \end{enumerate}
\end{lemma}

\section{Covering spaces}
\begin{definition}[Covering map]
  A map $p:X'\rightarrow X$ is a \emph{covering map} if $\forall x\in X$, there exists
  $U$ and open neighbourhood of $x$, and a discrete set $\Delta$ and a homeomorphism 
  $h_U:U\times\Delta\rightarrow p^{-1}(U)$ such that \[p\circ h_u=\pi_U:U\times\Delta\rightarrow 
  U\] and such a neighbourhood $U$ is called a \emph{covering neighbourhood}.
\end{definition}

\begin{definition}[Lift]
  Let $f:Y\rightarrow X$ and $g:Z\rightarrow X$ be two maps,
  \[
    \begin{tikzcd}
      &Z\arrow{d}{g} \\ Y\arrow{ur}{\tilde{f}} \arrow{r}{f}&X
    \end{tikzcd}\]
  a lift of $f$ is a map $\tilde{f}:Y\rightarrow Z$ such that \[g\circ\tilde{f}=f\]
\end{definition}

\subsection{Path/Homotopy lifting lemma}
\begin{lemma}[Path/Homotopy lifting lemma]
  Let $p:X'\rightarrow X$ be a covering map and $f:I^n\rightarrow X$ a continuous map.
  Then for any $x'\in p^{-1}(f(U))$, there exists a unique lift $\tilde{f}$ of $f$ to 
  $X'$, where $\tilde{f}(0)=x'$. 
\end{lemma}

\begin{definition}
  A covering space $p:X'\rightarrow X$ is \emph{trivial} if $X$ is a covering neighbourhood.
\end{definition}

\begin{lemma}
  Suppose $p:X'\rightarrow X$ is a trivial covering map, and $f:Y\rightarrow X$ is continuous, 
  $Y$ connected, the for any $y_0\in Y$ and $x'\in p^{-1}(f(y_0))$, there exists a unique 
  lift $\tilde{f}:Y\rightarrow X'$ such that $\tilde{f}(y_0)=x'$.
\end{lemma}

\begin{lemma}
  Let $X$ be a compact metric space. Then a continuous function $f:X\rightarrow\mathbb{R}$  attains a maximum and minimum value on $X$.
\end{lemma}

\begin{lemma}[Lebesgue's number lemma]
  Let $X$ be a compact metric space, $\mathcal{U}$ an open cover of $X$, then there 
  exists $\epsilon>0$ such that for all $x\in X$, there exists $U\in\mathcal{U}$ such that 
  $B_\epsilon(x)\subseteq U$. \\
  Such an $\epsilon$ is called the Lebesgue number for $\mathcal{U}$.
\end{lemma}

\begin{lemma}[(+)]
  Let $p:X'\rightarrow X$ be a covering space, and $f:Y\rightarrow X$ a continuous map, $Y$ 
  connected. Then two lifts $\tilde{f_1},\tilde{f_2}:Y\rightarrow X'$ are equal for all 
  $y\in Y$ if and only if they are equal for some $y\in Y$.
\end{lemma}

\begin{corollary}
  If $[\gamma]\in\pi_1(X,x_0)$ and there exists a covering space $X'$ of $X$ so that 
  $\gamma$ lifts to a non-closed path then $[\gamma]\neq1\in\pi_1(X,x_0)$.
\end{corollary}

\begin{corollary}
  \[\pi_1(S^1)\neq1\]
\end{corollary}

\begin{corollary}
  $\text{id}_{S^1}:S^1\rightarrow S^1$ is \emph{not} null-homotopic. In particular $S^1$
  is not contractible.
\end{corollary}

\subsection{Winding numbers}
\begin{definition}
  Let $\gamma$ be a closed path in $S^1$. The \emph{winding number} of $\gamma$, 
  $\omega(\gamma)$ is the integer $\tilde{\gamma(1)}-\tilde{\gamma(0)}$ where $\tilde{\gamma}$
  is any lift of $\gamma$ to $\mathbb{R}$.
\end{definition}

\begin{proposition}
  $\omega(\gamma)$ is well-defined, and only depends on the free homotopy classes of $\gamma$.
\end{proposition}

\begin{proposition}
  If $\gamma\simeq\gamma'$ (freely homotopic) then $\omega(\gamma)=\omega(\gamma')$.
\end{proposition}

\subsection{Covering transformations}

\begin{definition}[Covering transformation]
  Let$p:X'\rightarrow X$ be a covering map. A \emph{covering transformation} is a homeomorphism $h:X'\rightarrow X'$ such that $p\circ h=p$
\end{definition}

\begin{theorem}
  If $X'$ is the universal cover of space $X$, then 
  \[\pi_1(X,x_0)=\{h:X'\rightarrow X'\,|\,p\circ h =p\}\]
\end{theorem}

\end{document}