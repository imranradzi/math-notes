\documentclass[a4paper,14pt]{extarticle}
\usepackage{color}
\usepackage{amsmath}
\usepackage{amsthm}
\usepackage{amssymb}
\usepackage{bbold}
\usepackage{mathtools}
\usepackage{centernot}
\usepackage{tikz}
\usepackage{tikz-cd}
\usepackage{caption}
\usepackage{adjustbox}
\usepackage{multicol}  

\theoremstyle{definition}
\newtheorem*{theorem}{Theorem}
\newtheorem*{definition}{Definition}
\newtheorem*{lemma}{Lemma}
\newtheorem*{corollary}{Corollary}
\newtheorem*{proposition}{Proposition}
\newtheorem*{eg}{Example}
\newtheorem*{remark}{Remark}


\begin{document}

\title{\textbf{Topology and Groups - MATH0074}}
\author{\textbf{Based on lectures by Dr. Lars Louder}\\ Notes taken by Imran Radzi}
\date{}
\maketitle

\pagenumbering{roman}
(Revision) notes based on the Autumn 2021 Topology and Groups lectures by Dr. Lars Louder.
\begingroup
\let\cleardoublepage\clearpage
\tableofcontents
\endgroup
\newpage
\pagenumbering{arabic}

\section{Point-set Topology}

\subsection{Preliminaries}
\begin{definition}[Topological space]
  A topological space is a pair $(X,\mathcal{T})$ such that
  \begin{enumerate}
    \item $X$ is a set 
    \item $\mathcal{T}\subset \mathcal{P}(X)$ is a collection of subsets of $X$
    \item $\emptyset\in\mathcal{T}, \,X\in\mathcal{T}$
    \item $\mathcal{T}$ is closed under finite intersections and arbitrary unions
  \end{enumerate}
\end{definition}

\begin{definition}[Open neighbourhood]
  If $x\in X, \,U$ open in $X$, and $x\in U$, then $U$ is an \emph{open neighbourhood} of $x$.
\end{definition}

\begin{definition}[Hausdorff spaces]
  A topological space $(X,\mathcal{T})$ is \emph{Hausdorff} if $\forall \,x,y\in X$, there 
  exists $U,V$ open neighbourhoods of $x,y$ respectively such that $U\cap V=\emptyset$.
\end{definition}

\begin{definition}[Homeomorphisms]
  A map $f:X\rightarrow Y$ is a \emph{homeomorphism} if 
  \begin{enumerate}
    \item $f$ is bijective
    \item $f$ is continuous
    \item $f^{-1}$ is continuous
  \end{enumerate}
\end{definition}

\begin{definition}[Continuous maps]
  A map $f:X\rightarrow Y$ is continuous if $\forall \,U\text{ (open)}\subset Y$,
  $f^{-1}(U)$ is open in $X$.
\end{definition}

\begin{definition}
  If $\mathcal{T}$ and $\mathcal{T}'$ are topologies on $X$ such that $\mathcal{T}\subsetneq\mathcal{T}'$ then $\mathcal{T}'$ is \emph{finer} than $\mathcal{T}$, and 
  $\mathcal{T}$ is \emph{coarser} than $\mathcal{T}'$.
\end{definition}

\begin{proposition}
  $\text{id}:(X,\mathcal{T}\rightarrow (X,\mathcal{T}')$ is continuous if and only if 
  $\mathcal{T}$ is finer than $\mathcal{T}'$.
\end{proposition}

\begin{definition}[Subspace topology]
  If $X$ is a topological space, $Y\subset X$, the subspace topology on $Y$ is defined by
  \[U\text{ open in } Y\iff \exists V\text{ open in }X \text{ such that }U=Y\cap V\]
\end{definition}

\begin{definition}
  If a map $f:X\rightarrow Y$ is continuous, the \emph{image} of $f$ is the set 
  \[f(X)=\{f(x)\,|\,x\in X\}\subset Y\] with the subspace topology.
\end{definition}

\begin{definition}[Product topology]
  Let $X, Y$ be spaces. The \emph{product topology} on $X\times Y$ is the smallest (coarsest)
  topology making the projections
  \[p_X:X\times Y\rightarrow X, \,\,\,p_Y:X\times Y\rightarrow Y\]
  continuous.
\end{definition}

\begin{proposition}
  Product of Hausdorff spaces if Hausdorff.
\end{proposition}

\subsection{Connectedness}
\begin{definition}[Connectedness]
  A space $X$ is \emph{disconnected} if there exists a surjective continuous map 
  $f:X\rightarrow\{p_1,p_2\}$. A space is \emph{connected} if every continuous function
  $f:X\rightarrow\{p_1,p_2\}$ is constant.
\end{definition}

\begin{definition}
  A pair of sets $U,V\subset X$ is said to disconnect $X$ if they are non-empty, disjoint,
  $U\cup V=X$ and both are open.
\end{definition}

\begin{definition}
  $X$ is disconnected if there exists $U,V$ which disconnect $X$.
\end{definition}


\begin{definition}[Path]
  A \emph{path} in $X$ is a continuous map $\gamma:[0,1]\rightarrow X$. $\gamma$ is a path 
  from $\gamma(0)$ to $\gamma(1)$. $a,b\in X$ are said to be connected by a path if there 
  is a path from $a$ to $b$.
\end{definition}

\begin{definition}[Path-connectedness]
  A space $X$ is \emph{path-connected} if for all $x,y$, there exists
  \[\gamma:[0,1]\rightarrow X\text{ such that }\gamma(0)=x, \,\gamma(1)=y\]
\end{definition}

\noindent or equivalently,

\begin{definition}
  We say $X$ is path-connected if there exists a unique equivalence class, where the 
  equivalence relation $\sim$ is defined $a\sim b$ if and only if there exists a path from 
  $a$ to $b$.
\end{definition}

\begin{proposition}
  Suppose $X$ is connected. Then, if $f:X\rightarrow Y$, then $f(X)\subset Y$ is connected.
\end{proposition}

\begin{proposition}
  $[0,1]$ is connected.
\end{proposition}

\begin{corollary}
  If $X$ is path-connected, then $X$ is connected.
\end{corollary}

\begin{definition} $X\subset\mathbb{R}$ is an \emph{interval} if $a\leq b\leq c$, 
  $a,c\in X\implies b\in X$. 
\end{definition}

\begin{proposition}
  A subset of $\mathbb{R}$ is connected if and only if it is an interval.
\end{proposition}

\begin{definition}[Locally (path) connected]
  A space $X$ is locally (path ) connected at a point $p$ if for every open neighbourhood $U$ of $p$, 
  there exists a (path) connected open neighbourhood $V$ of $p$ such that $p\in V\subset U$.
\end{definition}

\begin{proposition}
  If $X$ is locally path-connected then the path components of $X$ are open.  
\end{proposition}

\begin{proposition}
  If $X$ is connected and locally path-connected, then $X$ is path connected.
\end{proposition}


\subsection{Compactness}
\begin{definition}[Open cover]
  An \emph{open cover} of a space $X$ is a collection of open sets $\mathcal{U}$ such that
  \[X=\bigcup_{U\in\mathcal{U}}U\]
\end{definition}

\begin{definition}
  A space $X$ is \emph{compact} if every open cover has a finite subcover.
\end{definition}

\begin{lemma}
  Closed subset sof compact spaces are compact.
\end{lemma}

\begin{theorem}
  If $X,Y$ are compact, then $X\times Y$ is compact.
\end{theorem}

\begin{theorem}[Heine-Borel theorem]
  $X\subset\mathbb{R}^n$ is compact if and only if $X$ is closed and bounded.
\end{theorem}

\begin{theorem}
  $[0,1]$ is compact.
\end{theorem}

\begin{theorem}
  If $f:X\rightarrow Y$ is continuous, $X$ compact, then $f(X)\subset Y$ is compact with respect to the subspace topology.
\end{theorem}

\begin{proposition}
  If $C\subset Y$ is compact, $Y$ Hausdorff, then $C$ is closed.
\end{proposition}

\begin{proposition}
  If $f:X\rightarrow Y$ is a continuous bijection, $X$ compact, $Y$ Hausdorff, then $f$ 
  is a homeomorphism
\end{proposition}

\subsection{Quotient spaces}
\begin{definition}[Quotient map]
  Let $q:X\rightarrow Y$ be a continuous surjection. Then $q$ is a \emph{quotient map}
  if $q^{-1}(Y)$ is open if and only if $U$ is open. (A bijective quotient map is a 
  homeomorphism)
\end{definition}

\begin{definition}[Quotient space]
  Let $X$ be a space, and $\sim$ an equivalence relation on $X$, and 
  $q:X\rightarrow X/\sim = Y$ the quotient map. The quotient topology on $Y$ is defined by
  $U$ open in $Y$ if and only if $q^{-1}(U)$ is open in $X$.
\end{definition}

\begin{lemma}
  \[
  \begin{tikzcd}
    X \arrow{r}{f} \arrow[swap]{dr}{q} & Z  \\
     & Y \arrow{u}{h}
  \end{tikzcd}
\]
  Let $f$ be continuous, and suppose $f$ factors through $:X\rightarrow Y$, a quotient 
  map, i.e., $\exists h:Y\rightarrow Z$ such that $h\circ q = f$. Then $h$ is continuous.
\end{lemma}

\begin{proposition}
  Let $f:X\rightarrow Y$ be a continuous surjection with $X$ compact, $Y$ Hausdorff. Then 
  $f$ is a quotient map. 
\end{proposition}

\begin{definition}[Disjoint union]
  Let $X_1,X_2$ be topological spaces. The \emph{disjoint union} of $X_1$ and $X_2$,
  $X_1\sqcup X_2$ is the space with the underlying set $X_1\sqcup X_2$, with $U$ open 
  in $X_1\sqcup X_2$ if and only if $U\cap X_1$ is open in $X_1$, and $U\cap X_2$ is 
  open in $X_2$.
\end{definition}

\begin{definition}[Cell complex]
  A \emph{cell complex} is a space built up inductively, as follows
  \begin{enumerate}
    \item ($n=0$) We start with a discrete set $X^{(0)}$ consisting of points, which we call 
    $0$-cells $\{e_i^0\,|\,i\in I_0\}, \,e_i^0\cong pt$. $X^{(0)}=\bigsqcup_i e_i^0$ is called the $0$-skeleton.
    \item ($n>0$) We add a (possibly empty) subset of $n$-cells $\{e_i^n\,|\,i\in I_n\}$ 
    $e_i^n\cong D^n$, the $n$-dimensional disk, and a continuous map 
    \[\phi_i^n:\partial e_i^n\cong S^{n-1}\rightarrow X^{(n-1)}\] and here the $n$-skeleton 
    is \[X^{(n)}=X^{(n-1)}\sqcup\bigsqcup e_i^n/\sim\]
  \end{enumerate}
  A space $X$ is a cell complex if there exists $X^{(0)}\subset X^{(1)}\subset\ldots$ 
  as above, with the condition that $U$ is open in $X$ if and only if 
  $X^{(n)}\cap U$ is open for all $n$. \\

  $X^{(0)}\subseteq X^{(1)}\subseteq\ldots$ is called the \emph{cell decomposition} of $X$.
\end{definition}



















\end{document}