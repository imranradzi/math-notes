\documentclass[a4paper,14pt]{extarticle}
\usepackage{color}
\usepackage{amsmath}
\usepackage{amsthm}
\usepackage{amssymb}
\usepackage{centernot}
\usepackage{tikz}
\usepackage{caption}
\usepackage{tikz-cd}

\theoremstyle{definition}
\newtheorem*{theorem}{Theorem}
\newtheorem*{definition}{Definition}
\newtheorem*{lemma}{Lemma}
\newtheorem*{proposition}{Proposition}
\newtheorem*{eg}{Example}

\begin{document}


\title{\textbf{Topology and Groups - MATH0074}}
\author{\textbf{Based on lectures by Dr Lars Louder}\\ Notes taken by Imran Radzi}
\date{}
\maketitle

\pagenumbering{roman}
Notes based on the Autumn 2021 Topology and Groups lectures by Dr Lars Louder.

\begingroup
\let\cleardoublepage\clearpage
\tableofcontents
\endgroup
\newpage
\pagenumbering{arabic}

\section{Review of metric topology}

A metric space is a pair $(X, d)$ where $X$ is a set and $d$ is a metric on $X$. Recall the notions of an open and closed ball around a point $x\in X$
\[B_\epsilon(x)=\{x'\in X:d(x,x')<\epsilon\}\] \[\overline{B_\epsilon(x)}=\{x'\in X:d(x,x')<\epsilon\}\]

\begin{definition}
	A map $f:X\to Y$ ($X,Y$ metric space) is continuous at $x\in X$, if $\forall\epsilon>0,\,\exists\delta>0$ such that \[f(B_\delta(x))\subseteq B_\epsilon(f(x))\]
\end{definition}
We say $f$ is continuous if it is continuous at every point in $X$. \\

If $X$ is a metric space then $U\subseteq X$ is said to be \emph{open} if \[\forall x\in U,\,\exists\epsilon>0\text{ such that } B_\epsilon(x)\subseteq U\]

\section{Topological spaces}
\begin{definition}[Topological space]
	A \emph{topological space} is a pair $(X,\mathcal{T})$, where $X$ is a set and $T\subseteq 2^X (=\mathcal{P}(X))$ satisfying the following
	\begin{enumerate}
		\item $\phi,\,X\in\mathcal{T}$
		\item $\mathcal{T}$ is closed under finite intersections and closed under arbitrary unions
	\end{enumerate}
\end{definition}

\begin{eg} \hfill
	\begin{enumerate}
		\item A metric space $(X,d)$ can be considered as a topological space in its own right with the topology $\mathcal{T}$ (called the metric topology) on $X$ being the collection of 
			open sets determined by $d$.
		\item $(X,2^X)$, the topology on $X$ being the collection of all subsets of $X$ is called the discrete topology.
		\item $(X,\{\emptyset, X\}$, with the topology on $X$ consisting of only $\phi$ and $X$ is called the indiscrete topology.
		\item $(X,\{U\subseteq X:|X\setminus U|<\infty\textit{ or }U=\emptyset\})$, the cofinite topology on $X$.
	\end{enumerate}
\end{eg}

\begin{definition}
	If $(X,\mathcal{T})$ is a topological space, $x\in X, \,U\in\mathcal{T}$ then we say $U$ is an \emph{open neighbourhood} of $x$ if $x\in U$.
\end{definition}

\begin{definition}[Hausdorff space]
	A topological space is \emph{Hausdorff} if given $x,y\in X, \,x\neq y, \,\exists U,V\in\mathcal{T}$ such that \[U\cap V=\emptyset\]
\end{definition}

\begin{definition}[Continuous map]
	A map $f:X\to Y$ is \emph{continuous} if $f^{-1}(U)$ is open in $X$, for $U$ open in $Y$, i.e., the preimage of every open set is open.
\end{definition}

\begin{definition}[Homeomorphism]
	A continuous map $f:X\rightarrow Y$ is a \emph{homeomorphism} if
	\begin{enumerate}
		\item $f$ is a bijection
		\item $f^{-1}:Y\rightarrow X$ is continuous
	\end{enumerate}
\end{definition}

We say that a property of a space is topological if it is preserved under homeomorphism.



\end{document}



































