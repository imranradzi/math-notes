\documentclass[a4paper,14pt]{extarticle}
\usepackage{color}
\usepackage{amsmath}
\usepackage{amsthm}
\usepackage{amssymb}
\usepackage{bbold}
\usepackage{commath}
\usepackage{bbm}
\usepackage{mathtools}
\usepackage{centernot}
\usepackage{tikz}
\usepackage{tikz-cd}
\usepackage{caption}
\usepackage{adjustbox}
\usepackage{multicol}  

\theoremstyle{definition}
\newtheorem{theorem}{Theorem}
\newtheorem{definition}{Definition}
\newtheorem{lemma}{Lemma}
\newtheorem{corollary}{Corollary}
\newtheorem{proposition}{Proposition}
\newtheorem{eg}{Example}
\newtheorem{remark}{Remark}


\begin{document}
\title{\textbf{Measure Theory - MATH0017}}
\author{}
\date{}
\maketitle

\pagenumbering{roman}
\begingroup
\let\cleardoublepage\clearpage
\tableofcontents
\endgroup
\newpage
\pagenumbering{arabic}

\section{Preliminaries}
\begin{definition}
  \hfill
  \begin{enumerate}
    \item $f:X\rightarrow Y$ is an injective map if and only if $f(x)=f(y)\implies x= y$.
    \item $f:X\rightarrow Y$ is a surjective map if and only if $\{f(x)\mid x\in X\}$ is the whole of $Y$.
    \item $f:X\rightarrow Y$ is a bijective map if and only if it is both injective and surjective.
  \end{enumerate}
\end{definition}
\begin{definition}
  Given two (finite or infinite) sets 
  $X$ and $Y$, if there exists an 
  injective map $f:X\rightarrow Y$ then
  we say that the cardinality of $X$ 
  is less than or equal to the cardinality of $Y$ (and in this case
  we write $\#X\leq\#Y)$.
\end{definition}
\begin{definition}
  If $\#X\leq\#Y$ and $\#Y\leq\#X$ then 
  we say that $X$ and $Y$ have the same
  cardinality (and in this case we 
  write $\#X=\#Y$).
\end{definition}
\begin{definition}
  A set $X$ is called countable if 
  $\#X\leq \#\mathbb{N}$.
\end{definition}
\begin{definition}
  The cardinality of $(0,1)$ is called 
  continuum.
\end{definition}
\begin{definition}
  A set $D\subset\mathbb{R}$ has (Lebesgue) measure zero if for every
  $\epsilon>0$ there is a countable 
  collection of open intervals $I_j=
  (a_j,b_j)$ such that $D\subset\cup_{j=1}^\infty I_j$ and $\sum_{j=1}^\infty |I_j|<\epsilon$, where 
  $|I_j|=b_j-a_j$ denotes the length of 
  the interval $I_j$.
\end{definition}
\newpage
\section{$\sigma$-algebras and measures}
\begin{definition}
  A $\sigma$-algebra on $\mathcal{A}$ on a set 
  $X$ is a family of subsets of $X$ with 
  the following properties:
  \begin{enumerate}
    \item $X\in\mathcal{A}$
    \item \textbf{(closedness under complementation)}\\ $A\in\mathcal{A}\implies X\setminus A\in\mathcal{A}$
    \item \textbf{(closedness under countable unions)}\\ $\{A_j\}_{j\in\mathbb{N}}\subset A\implies
    \cup_{j\in\mathbb{N}}\in\mathcal{A}$
  \end{enumerate}
\end{definition}
\begin{definition}
  A (positive) measure $\mu$ on a space $X$ is 
  an assignment \[\mu:A\rightarrow[0,+\infty]\]
  defined on a $\sigma$-algebra $\mathcal{A}$,
  such that 
  \begin{enumerate}
    \item $\mu(\emptyset)=0$
    \item \textbf{($\sigma$-additivity)} $\{A_j\}_{j=1}^\infty\subset\mathcal{A}, A_j$ pairwise disjoint $\implies \mu(\cup_{j=1}^\infty A_j)=
    \sum_{j=1}^\infty \mu(A_j)$
  \end{enumerate}
  The triple $(X,\mathcal{A},\mu)$ is called a 
  \textit{measure space}. On the other hand 
  $(X,\mathcal{A})$ is called a \textit{measurable space}.
\end{definition}
\textbf{Other properties of $\sigma$-algebras and measures.}
\begin{enumerate}
  \item Let $A\subset X,B\subset X$. If $A,B\in\mathcal{A}$ then $B\setminus A\in\mathcal{A}$
  \item Let $A\in\mathcal{A}$ with $\mu(A)$ finite. We then also have $\mu(X\setminus A)=\mu(X)-\mu(A)$.
  \item Let $A,B\in\mathcal{A}, A\subset B$. Then 
  $\mu(A)\leq \mu(B)$.
\end{enumerate}
\begin{proposition}
  Whenever $\{\mathcal{A}_j\}_{j\in J}$ is a family of $\sigma$-algebras in $X$, then the intersection
  $\cap_{j\in J}\mathcal{A}_j$ is again a 
  $\sigma$-algebra in $X$.
\end{proposition}
\begin{definition}
  Given any family $\mathcal{G}\subset\mathcal{P}(X)$ (i.e., any family of subsets of $X$) there exists a 'smallest' $\sigma$-algebra containing 
  $\mathcal{G}$. This is called the $\sigma$-algebra
  generated by $\mathcal{G}$ and denoted by $\sigma(\mathcal{G})$.
\end{definition}
\newpage
\section{Outer measures}
\begin{definition}
  An outer measure $\mu$ on a space $X$ is an assignment (defined on all subsets of $X$)
  $\mu:\mathcal{P}(X)\rightarrow[0,+\infty]$ such 
  that 
  \begin{enumerate}
    \item $\mu(\emptyset)=0$
    \item \textbf{(monotonicity)} $A\subset B\implies \mu(A)\leq\mu(B)$
    \item \textbf{($\sigma$-subadditivity)} $A_j\subset X, \{A_j\}_{j=1}^\infty\implies\mu(\cup_{j=1}^\infty A_j)\leq\sum_{j=1}^\infty\mu(A_j)$
  \end{enumerate}
\end{definition}
2. and 3. can equivalently be phrased as 
\[A\subset\cup_{j=1}^\infty A_j\implies\mu(A)\leq
\sum_{j=1}^\infty\mu(A_j)\]
\begin{definition}
  Let $\mu$ be an outer measure on a space $X$. A 
  set $A\subset X$ is said to be $\mu$-measurable if for every $B\subset X$ we have 
  \[\mu(B)=\mu(B\cap A)+\mu(B\setminus A)\]
\end{definition}
\noindent\textbf{Theorem 3.1.1.} Let $\mu$ be an outer measure on a space $X$. The family 
\[\Sigma=\{A\subset X\mid A\text{ is $\mu$-measurable}\}\]
is a $\sigma$-algebra on $X$.
\begin{proposition}
  Let $\mu$ be an outer measure on $X$ and let $\Sigma$ be the $\sigma$-algebra of $\mu$-measurable sets. Then whenever $\{A_k\}_{k=1}^\infty\subset\Sigma$ and $A_k$ are pairwise 
  disjoint we have $\mu(\cup_{k=1}^\infty A_k)=
  \sum_{k=1}^\infty \mu(A_k)$.
\end{proposition}
\begin{proposition}
  Let $\mu$ be an outer measure on $X$ and let 
  $A\subset X$ be such that $\mu(A)=0$. Then $A$ is 
  $\mu$-measurable.
\end{proposition}
\begin{definition}
  A measure space $(X,\mathcal{A},\mu)$ is complete if, whenever $D\in\mathcal{A}$ is such that $\mu(D)=0$, and $N\subset D$, then $N\in\mathcal{A}$ (it follows that $\mu(N)=0$ as well).
\end{definition}
\begin{definition}
  We say that $\mathcal{S}\subset\mathcal{P}(X)$
  is a 
  covering class (for $X$) when the following two 
  properties hold:
  \begin{enumerate}
    \item $\emptyset\in\mathcal{S}$
    \item for any $A\subset X$ there exists a countable collection of sets in $\mathcal{S}$
    that 'covers $A$', i.e., for any $A\subset X$,
    there exists $\{S_j\}_{j=1}^\infty$ such that 
    $S_j\in\mathcal{S}$ for every $j$ and 
    $A\subset\cup_{j=1}^\infty S_j$. (Note that 
    for a given $A$ there might be plenty of choices of $\{S_j\}_{j=1}^\infty$ that cover $A$. Each such choice $\{S_j\}_{j=1}^\infty$ is a possible 'cover', or 'covering', of $A$.)
  \end{enumerate}
\end{definition}
\noindent\textbf{Theorem 3.2.1.} Let $\mathcal{S}\subset\mathcal{P}(X)$ be a covering class of $X$ and $\lambda:\mathcal{S}\rightarrow[0,\infty]$ be given, with $\lambda(\emptyset)=0$. Then, the 
assignment $\mu:\mathcal{P}(X)\rightarrow[0,+\infty]$ defined by 
\[\mu(A):=\inf\bigl\{\sum_{j=1}^\infty\lambda(S_j)\mid S_j\in\mathcal{S}\text{ and }A\subset\cup_{j=1}^\infty S_j\bigr\}\] is an outer measure on $X$.
\newpage
\section{From pre-measures to outer measures - Lebesgue measure in $\mathbb{R}^n$}
\begin{definition}
  A collection $\mathcal{S}\subset\mathcal{P}(X)$
  is a semi-ring (or semi-algebra) on $X$ if 
  it satisfies the three properties below
  \begin{enumerate}
    \item $\mathcal{S}$ contains $\emptyset$
    \item if $R_1,R_2\in\mathcal{S}$, then $R_1\cap R_2\in\mathcal{S}$
    \item if $R_1,R_2\in\mathcal{S}$, then there exists finitely many disjoint $T_1,\ldots,T_S
    \subset\mathcal{S}$ such that $R_1\setminus R_2=\cup_{s=1}^S T_s$
  \end{enumerate}
\end{definition}
\noindent We denote the set of half-open rectangles in $\mathbb{R}^n$ by $\mathcal{I}^n$.
\begin{definition}
  Let $\mathcal{S}\subset\mathcal{P}(X)$ be a 
  semi-ring. An assignment $\lambda:\mathcal{S}\rightarrow[0,+\infty]$ is called a 
  pre-measure if 
  \begin{enumerate}
    \item $\lambda(\emptyset)=0$
    \item if $\{R_j\}_{j=1}^\infty\subset\mathcal{S}, R_j\cap R_k=\emptyset$ when $j\neq k$,
     $\cup_{j=1}^\infty R_j\in\mathcal{S}$ then $\lambda(\cup_{j=1}^\infty R_j)=\sum_{j=1}^\infty\lambda(R_j)$
  \end{enumerate}
\end{definition}
\noindent\textbf{Theorem 4.2.1.} Let $\mathcal{S}=\mathcal{I}^n$, $\lambda$ be defined as 
\[\lambda([a,b))=\Pi_{j=1}^n(b_j-a_j)\,\,\,\,\,\,\,\,\,\,\,(\star)\] and let 
$\mu$ be the outer measure constructed from $\mathcal{S}$ and $\lambda$ in Theorem 3.2.1. Then (i) $\lambda$ and $\mu$ agree on $\mathcal{I}^n$ and moreover (ii) every $\mathcal{R}\in\mathcal{I}^n$ is $\mu$-measurable.

\vspace{12pt}

\noindent\textbf{Theorem 4.2.2} Let $\mathcal{S}\subset\mathcal{P}(X)$ be a covering class and a semi-ring and let $\lambda:\mathcal{S}\rightarrow[0,+\infty]$ be a pre-measure. Denote by $\mu$ the outer measure constructed from $\mathcal{S}$ and $\lambda$ in Theorem 3.2.1. Then $\lambda$ and $\mu$ agree on $\mathcal{S}$ and every $S\in\mathcal{S}$ is $\mu$-measurable.

\vspace{12pt}

\noindent\textbf{Lebesgue measure:} We have 
obtained, in Theorem 4.2.1 an outer measure $\mu$ that extends the assignment $\lambda$ 
defined in $(\star)$ and such that half-open rectangles are $\mu$-measurable. Restricting the action of $\mu$ to the $\sigma$-algebra of the 
$\mu$-measurable sets gives rise to the \textit{n-dimensional Lebesgue measure} on $\mathbb{R}^n$, that we denote by $\lambda^n$.
\begin{proposition}
  The $\sigma$-algebra generated by $\mathcal{I}^n$ is the same as the Borel $\sigma$-algebra
  $\mathcal{B}(\mathbb{R}^n)$ (i.e., the $\sigma$-algebra generated by open sets of $\mathbb{R}^n$). This implies in particular that $\mathcal{B}(\mathbb{R}^n)\subset\Sigma_n$, which is usually phrased by saying that every Borel set (i.e. every element of $\mathcal{B}(\mathbb{R}^n)$) is 
  Lebesgue-measurable.
\end{proposition}
\noindent\textbf{Lemma 4.3.1} Every open set in $\mathbb{R}^n$ is the countable union of half-open rectangles. 

\vspace{12pt}

\noindent For $y\in\mathbb{R}^n$ and $Z\subset\mathbb{R}^n$ we write $y+Z:=\{x\in\mathbb{R}^n\mid x=y+z\text{ for some }z\in Z\}$ (the translation of $Z$ by the vector $y$).
\begin{proposition}
  The Lebesgue measure is translation invariant, namely, whenever $Z\in\Sigma_n$ then $y+Z\in\Sigma_n$ for every $y\in\mathbb{R}^n$ and moreover $\lambda^n(y+Z)=\lambda^n(Z)$
\end{proposition}
\noindent\textbf{Theorem 4.4.1} Let $\mathcal{S}$ be a semi-algebra on $X$ and $\lambda$ a pre-measure defined on $\mathcal{S}$. Assume that there exists a countable collection of $R_j$ in $\mathcal{S}$ such that $R_j\subset R_{j+1}$ for all $j$ and $\cup_{j=1}^\infty R_j=X$ ( this is called an exhausting sequence) with $\lambda(R_j)<\infty$ for all $j\in\mathbb{N}$. Let $\mu$ be the outer measure defined through
Carathéodory's Theorem
and $\Sigma$ the $\sigma$-algebra of the $\mu$-measurable sets. \\
Given any outer measure $\tilde{\mu}:\mathcal{P}(X)\rightarrow[0,\infty]$ with the property that 
$\tilde{\mu}=\lambda$ on $\mathcal{S}$, we have necessarily that $\tilde{\mu}=\mu$ on $\Sigma$.
\begin{definition}
  A measure space $(X,\mathcal{A},\mu)$ is said to be $\sigma$-finite if there exists $A_j\in\mathcal{A}$ such that $A_j\subset A_{j+1}$ for all $j$, $\cup_{j=1}^\infty A_j=X$ and $\mu(A_j)<\infty$ for all $j\in\mathbb{N}$. (By abuse of language, the measure $\mu$ in this case is also called $\sigma$-finite.) If $\mu(X)<\infty$ then we say that $\mu$ is a 
  finite measure on $X$ and $(X,\mathcal{A},\mu)$ is a finite measure space.
\end{definition}
\newpage
\section{Hausdorff measures}
Let $A\subset\mathbb{R}^n$ be arbitrary. For fixed $\alpha\geq0$ and $\delta>0$ we consider all possible coverings of $A$ using a collection of balls with radii $\leq\delta$. Set
\[\mathcal{H}_\delta^\alpha(A)=\inf\bigl\{\sum_{k=1}^\infty(r_k)^\alpha\mid A\subset\cup_{k=1}^\infty B_{r_k}(x_k), r_k\leq\delta\text{ for all }k\bigr\}\] We set 
\[\mathcal{H}^\alpha(A)=\lim\limits_{\delta\to0}\mathcal{H}_\delta^\alpha(A)=\sup\limits_{\delta>0}\mathcal{H}_\delta^\alpha(A)\]
\begin{proposition}
  For any $\alpha\geq0$, $\mathcal{H}^\alpha$ is an outer measure on $\mathbb{R}^n$.
\end{proposition}
\begin{proposition}
  For $0\leq\beta<\alpha <\infty$ and $A\subset\mathbb{R}^n$ it holds 
  \[\mathcal{H}^\beta(A)<\infty\implies\mathcal{H}^\alpha(A)=0\] \[\mathcal{H}^\alpha(A)>0\implies\mathcal{H}^\beta(A)=+\infty\]
\end{proposition}
\begin{definition}
  Given $A\subset\mathbb{R}^n$ we define 
  \[\dim_H(A)=\inf\{\alpha>0\mid\mathcal{H}^\alpha(A)=0\}\] to be the Hausdorff dimension of $A$. Note that $\dim_H(A)\in[0,n]$.
\end{definition}
\begin{proposition}
  For any $\alpha\geq0$ Borel sets are $\mathcal{H}^\alpha$-measurable.
\end{proposition}
\begin{definition}
  An outer measure $\mu$ on $\mathbb{R}^n$ is called 'metric' if whenever $A,B\subset\mathbb{R}^n$ are such that $dist(A,B)=\inf\{|x-y|\mid x\in A, y\in B\}>0$ (i.e. $A$ and $B$ are a positive distance apart) then $\mu(A\cup B)=\mu(A)+\mu(B)$.
\end{definition}
\begin{proposition}
  For $\alpha\geq0$, $\mathcal{H}^\alpha$ is a metric outer measure (on $\mathbb{R}^n$).
\end{proposition}
\begin{proposition}[Carathéodory criterium for Borel measures]
  Given a metric outer measure $\mu$, every Borel set is $\mu$-measurable.
\end{proposition}

\newpage
\section{Measurable functions}
\begin{definition}
  Let $(X,\mathcal{A})$ be a measurable space and consider $f:X\rightarrow\mathbb{R}$, where the target space $\mathbb{R}$ is implicitly endowed with the Borel $\sigma$-algebra. We say that $f$ is $\mathbb{A}$-measurable when $f^{-1}(A)\in\mathcal{A}$ whenever $A$ is open in $\mathbb{R}$. \\

  Given an outer measure $\mu$ on $X$ it is common to speak of $\mu$-measurable function $f:X\rightarrow\mathbb{R}$ with the implicit meaning that $X$ is equipped with the $\sigma$-algebra $\Sigma$ of $\mu$-measurable sets and $f$ is $\Sigma$-measurable.
\end{definition}
\begin{proposition}
  The collection $\{B\subset\mathbb{R}\mid f^{-1}(B)\in\mathcal{A}\}$ is a $\sigma$-algebra.
\end{proposition}
\begin{definition}
  Let $(X,\mathcal{A})$ be a measurable space and consider $f:X\rightarrow\overline{\mathbb{R}}$. We say that $f$ is $\mathcal{A}$-measurable when 
  \[f^{-1}(\{+\infty\})\in\mathcal{A}, f^{-1}(\{-\infty\})\in\mathcal{A}, f^{-1}(A)\in\mathcal{A}\text{ whenever $A$ is open in $\mathbb{R}$.}\]
  The condition \[f^{-1}(A)\in\mathcal{A}\text{ whenever $A$ is open in $\mathbb{R}$}\] in Definition 20 can be replaced by either of the following:
  \begin{enumerate}
    \item $f^{-1}(B)\in\mathcal{A}$ whenever $B\in\mathcal{B}(\mathbb{R})$
    \item $f^{-1}((a,+\infty))\in\mathcal{A}$ whenever $a\in\mathbb{R}$
    \item $f^{-1}((-\infty,a))\in\mathcal{A}$ whenever $a\in\mathbb{R}$.
  \end{enumerate}
\end{definition}
\begin{definition}
  A simple function $g:X\rightarrow\mathbb{R}$ on a measurable space $(X,\mathcal{A})$ is a function of the form 
  \[g(x)=\sum_{j=1}^M y_j\mathbbm{1}_{A_j}(x), y_j\in\mathbb{R}, A_j\text{ pairwise disjoint}, A_j\in\mathcal{A}\text{ for }j\in\{1,\ldots,M\}\]
\end{definition}
\noindent\textbf{Theorem 6.2.1.} Any $\mathcal{A}$-measurable function $f:X\rightarrow\overline{\mathbb{R}}$ on $(X,\mathcal{A})$ is the pointwise limit of simple functions $f_j$. Moreover if $f\geq0$ then we can choose $f_j\geq0$ and $f_j\leq f_{j+1}$ for all $j\in\mathbb{N}$ (so that $f=\sup_{j\in\mathbb{N}}f_j$). \\

When $f_j\rightarrow f$ pointwise and $f_j\leq f_{j+1}$ (increasing sequence) we write $f_j\uparrow f$

\vspace{12pt}

\noindent\textbf{Theorem 6.2.2.} Let $f,g:X\rightarrow\mathbb{R}$ be $\mathcal{A}$-measurable functions on $(X,\mathcal{A})$. Then the functions $f+g,f-g,fg$ and (when $g\neq0$) $f/g$ are $\mathcal{A}$-measurable. The same holds for the function $|f|$.

\vspace{12pt}

\noindent\textbf{Theorem 6.2.3.} Let $f_j:X\rightarrow\overline{\mathbb{R}}$ be $\mathcal{A}$-measurable functions on $(X,\mathcal{A})$ for $j\in\mathbb{N}$. Then the functions $\inf_j f_j,\sup_j f_j,\lim\sup_j f_j$ and $\lim\inf_j f_j$ are $\mathcal{A}$-measurable.

\vspace{12pt}

\noindent\textbf{Corollary 6.2.1.} Let $f,g:X\rightarrow\overline{\mathbb{R}}$ be $\mathcal{A}$-measurable functions on $(X,\mathcal{A})$. Then the functions $\min\{f,g\}$ and $\max\{f,g\}$ are $\mathcal{A}$-measurable.
\newpage
\section{Lebesgue integration}
\begin{definition}
  Let $(X,\mathcal{A},\mu)$ be a measure space. We say that $g:X\rightarrow\mathbb{R}$ is a simple function when $g$ admits a representation of the following form for some $M\in\mathbb{N}, y_j\in\mathbb{R}$ and $A_j\in\mathcal{A}$ (for $j=1,2,\ldots,M$) pairwise disjoint:
  \[g=\sum_{j=1}^M y_j\mathbbm{1}_{A_j}\]
  We write, in the case that $g\geq0, g\in\mathcal{E}^+$.
\end{definition}
\noindent\textbf{Lemma 7.1.1.} Let $g\geq0$ admit two distinct representation as a simple function, say $g=\sum_{j=1}^M y_j\mathbbm{1}_{A_j}=\sum_{k=1}^N x_k\mathbbm{1}_{B_k}$. Then \[\sum_{j=1}^M y_j\mu(A_j)=\sum_{k=1}^N x_k\mu(B_k)\]
\begin{definition}
  Given a measure space $(X,\mathcal{A},\mu)$ and $f:X\rightarrow\overline{\mathbb{R}}$ $\mathcal{A}$-measurable and non-negative, we define the integral of $f$ on $X$ with respect to $\mu$ as follows:
  \[\int_X f\,d\mu=\sup\{I_\mu(g)\mid g\in\mathcal{E}^+,g\leq f\}\in[0,+\infty]\]
\end{definition}
\noindent\textbf{Lemma 7.1.2.} Let $g\geq0$ be a simple function, say $g=\sum_{j=1}^M y_j\mathbbm{1}_{A_j}$. Then \[\int_X g\,d\mu=\sum_{j=1}^M y_j\mu(A_j)\]
\noindent\textit{Remark} 7.1.1.
\begin{enumerate}
  \item If $g,h\in\mathcal{E}^+$ then $g+h\in\mathcal{E}^+$ and $I_\mu(g+h)=I_\mu(g)+I_\mu(h)$
  \item If $g\in\mathcal{E}^+$ and $\lambda\in[0,\infty)$ then $\lambda g\in\mathcal{E}^+$ and $I_\mu(\lambda g)=\lambda I_\mu(g)$
\end{enumerate}
\begin{proposition}
  Let $f,g:X\rightarrow\overline{\mathbb{R}}$ be $\mathcal{A}$-measurable and non-negative functions on $(X,\mathcal{A},\mu)$. Assume that $f\geq g$. Then 
  $\in_X f\,d\mu\geq\int_X g\,d\mu$.
\end{proposition}
\noindent\textbf{Theorem 7.2.1.} (Fatou's lemma). Let $(X,\mathcal{A},\mu)$ be a measure space and consider a sequence $\{f_k\}_{k=1}^\infty$ of measurable functions $f_k:X\rightarrow\overline{\mathbb{R}}$ with $f_k\geq0$ for all $k$. Then 
\[\int_X (\lim\limits_{k\to\infty} f_k)\,d\mu\leq\lim\limits_{k\to\infty}\bigl(\int_X f_k \,d\mu\bigr)\]
\noindent\textbf{Theorem 7.2.2.} (Beppo Levi's theorem, or 'monotone convergence' theorem). Let $(X,\mathcal{A},\mu)$ be a measure space and consider a sequence $\{f_j\}_{j=1}^\infty$ of measurable functions $f_j:X\rightarrow\overline{\mathbb{R}}$ with $0\leq f_1\leq f_2\leq\ldots\leq f_m\leq f_{m+1}\leq\ldots$. Then 
\[\int_X (\sup_j f_j) \,d\mu=\sup_j(\int_X f_j \,d\mu)\] or equivalently, by Proposition 12, \[\int_X (\lim_j f_j) \,d\mu=\lim_j(\int_X f_j \,d\mu)\]
\noindent\textbf{Corollary 7.2.1.} Let $f_j:X\rightarrow\overline{\mathbb{R}}$ be an increasing sequence of non-negative $\mathcal{A}$-measurable functions, and let $f_j\uparrow f$ pointwise. Then $\int_X f_j \,d\mu\rightarrow \int_X f\,d\mu$ ($f:X\rightarrow\overline{\mathbb{R}}$ is $\mathcal{A}$-measurable by Theorem 6.2.3 and clearly $f\geq 0$).
\begin{proposition}
  Let $f,g:X\rightarrow\overline{\mathbb{R}}$ be measurable and non-negative on $(X,\mathcal{A},\mu)$, let $\alpha\in(0,+\infty)$. Then 
  \begin{enumerate}
    \item $\int_X (f+g) \,d\mu=\int_X f \,d\mu + \int_X g \,d\mu$
    \item $\int_X (\alpha f) \,d\mu=\alpha\int_X f \,d\mu$
  \end{enumerate}
\end{proposition}
\noindent\textbf{Corollary 7.2.2.} Let $u_j\geq0$ be measurable, $u_j:X\rightarrow\overline{\mathbb{R}}$. Then $\int_X (\sum_{j\in\mathbb{N}} u_j) \,d\mu=\sum_{j\in\mathbb{N}}\int_X u_j \,d\mu$.

\vspace{12pt}

\noindent\textbf{Theorem 7.2.3.} ('reverse Fatou's lemma' and 'dominated convergence' for non-negative functions). Let $u:X\rightarrow\overline{\mathbb{R}}$ be a non-negative $\mathcal{A}$-measurable function on the measure space $(X,\mathcal{A},\mu)$ with $\int_X u \,d\mu<\infty$. Let $f_j:X\rightarrow\mathbb{R}$ be a sequence of non-negative $\mathcal{A}$-measurable functions such that $f_j\leq u$ for all $j\in\mathbb{N}$. Assume that $f_j\rightarrow f$ pointwise, where $f:X\rightarrow\overline{\mathbb{R}}$ (this function is $\mathcal{A}$-measurable by Theorem 6.2.3.). Then 
\[\int_X f \,d\mu=\lim\limits_{j\to\infty} f_j \,d\mu\]
\begin{definition}
  Let $(X,\mathcal{A},\mu)$ be a measure space and $f:X\rightarrow\overline{\mathbb{R}}$ be a $\mathcal{A}$-measurable function. Whenever $\int_X f^+ \,d\mu$ and $\int_X f^{-1} \,d\mu$ are not both $+\infty$ we set 
  \[\int_X f \,d\mu:=\int_X f^+ \,d\mu - \int_X f^{-1} \,d\mu\] This value is in $[-\infty,+\infty]$.
\end{definition}
\begin{definition}[summable functions]
  With the same notations and conditions as in Definition 24, assume in addition that $\int_X f+ \,d\mu<\infty$ and $\int_X f^{-1} \,d\mu<\infty$. Then we say that $f$ is $\mu$-summable (note that in this case $\int_X f \,d\mu$ is finite).
\end{definition}
\textit{Remark} 7.3.1. Whenever $f:X\rightarrow\overline{\mathbb{R}}$ is $\mu$-measurable, the $\mu$-summability of $f$ is equivalent to the finiteness of $\int_X |f| \,d\mu$ (recall from Theorem 6.2.2 that the function $|f|$ is measurable if $f$ is measurable). This follows upon noticing that $|f|=f^+ + f^{-1}$ and so by Proposition 13(i) we have $\int_X |f| \,d\mu = \int_X f^+ \,d\mu + \int_X f^{-1} \,d\mu$.

\vspace{12pt}

\noindent\textbf{Theorem 7.3.1.} ((Lebesgue's) dominated convergence theorem (for signed functions) - weak statement). Let $u:X\rightarrow\overline{\mathbb{R}}$ be a non-negative $\mathcal{A}$-measurable function on the measure space $(X,\mathcal{A},\mu)$ with $\int_X u \,d\mu<\infty$ (i.e. $u$ is $\mu$-summable). Let $f_j:X\rightarrow\mathbb{R}$ be a sequence of $\mathcal{A}$-measurable functions usch that $|f_j|\leq u$ for all $j\in\mathbb{N}$. Assume that $f_j\rightarrow f$ pointwise (where $f:X\rightarrow\overline{\mathbb{R}}$). Then 
\[\int_X f \,d\mu=\lim\limits_{j\to\infty}\int_X f_j \,d\mu\]
\begin{proposition}
  Let $f,g:X\rightarrow\overline{\mathbb{R}}$ be $\mu$-summable, let $\alpha\in\mathbb{R}$. Then 
  \begin{enumerate}
    \item If $f+g$ is well-defined then it is $\mu$-summable and $\int_X(f+g) \,d\mu=\int_X f \,d\mu + \int_X g \,d\mu$
    \item $\alpha f$ is $\mu$-summable and $\int_X (\alpha f) \,d\mu=\alpha\int_X f \,d\mu$
  \end{enumerate}
\end{proposition}
\begin{proposition} Let $f,g:X\rightarrow\overline{\mathbb{R}}$ be $\mu$-summable. Then
  \begin{enumerate}
    \item $\max\{f,g\}$ and $\min\{f,g\}$ are $\mu$-summable
    \item $f\leq g\implies \int_X f \,d\mu\leq\int_X g \,d\mu$
    \item $|\int_X f \,d\mu|\leq\int_X|f| \,d\mu$
  \end{enumerate}
\end{proposition}
\begin{proposition}
  Let $(X,\mathcal{A},\mu)$ be a complete measure space, and let $f,g:X\rightarrow\overline{\mathbb{R}}$ be $\mu$-summable. Then $f+g$ is $\mu$-summable and we have $\int_X (f+g) \,d\mu=\int_X f \,d\mu + \int_X g \,d\mu$.
\end{proposition}
\noindent\textbf{Theorem 7.3.2} ((Lebesgue's) dominated convergence theorem - strong statement). Let $u:X\rightarrow\overline{\mathbb{R}}$ be a non-negative $\mathcal{A}$-measurable function on the complete measure space $(X,\mathcal{A},\mu)$ with $\int_X u \,d\mu<\infty$ (i.e. $u$ is $\mu$-summable). Let $f_j:X\rightarrow\overline{\mathbb{R}}$ be a sequence of $\mathcal{A}$-measurable functions such that $|f_j|\leq u$ for all $j\in\mathbb{N}$. Assume that $f_j\rightarrow f$ pointwise almost everywhere (that is, there exists $N\in\mathcal{A}$ such that $\mu(N)=0$ and $f_j(X)\rightarrow f(x)$ for all $x\in X\setminus N$), where $f:X\rightarrow\overline{\mathbb{R}}$. Then 
\[\lim\limits_{j\to\infty}\int_X |f-f_j| \,d\mu=0\]
\begin{definition}
  Given a measure space $(X,\mathcal{A},\mu)$, let $A\in\mathcal{A}$ and let $f:X\rightarrow\mathbb{R}$ be $\mathcal{A}$-measurable. Assume that $\mathbbm{1}_A f$ is $\mu$-integrable (note that $f\mathbbm{1}_A$ is automatically $\mathcal{A}$-measurable). Then the $\mu$-integral of $f$ on $A$ is defined as \[\int_A f \,d\mu=\int_X f\mathbbm{1}_A \,d\mu\] (if $f$ is assumed to be $\mu$-summable, then $\mathbbm{1}_A f$ is also $\mu$-summable, in particular it is $\mu$-integrable.)
\end{definition}
\begin{proposition}[Markov's inequality]
  Let $f:X\rightarrow\overline{\mathbb{R}}$ be $\mu$-summable on the measure space $(X,\mathcal{A},\mu)$. For every $c>0$ and $A\in\mathcal{A}$
  \[\mu(\{x\in X\mid|f(x)\geq c\}\cap A)\leq\frac{1}{c}\int_A |f| \,d\mu\] In particular choosing $A=X$ we have 
  \[\mu(\{x\in X\mid |f(x)|\geq c\})\leq \frac{1}{c}\int_X |f| \,d\mu\]
\end{proposition}
\begin{proposition}
  Let $I=[a,b]$ and let $f:I\rightarrow\mathbb{R}$ be a non-negative Riemann-integrable function with $\int_a^b f \,dx=R$. Denote with $\lambda$ the Lebesgue 1-dimensionaly measure on $\mathbb{R}$. Then $f$ is Lebesgue integrable and the two integrals yield the same value, i.e. $\int_I f \,d\lambda=R$.
\end{proposition}
\begin{proposition}
  Let $f:R\times[0,1]\rightarrow\mathbb{R}$ be such that 
  \begin{enumerate}
    \item for every $x\in\mathbb{R}$ the function $y\rightarrow f(x,y)$ is summable on $[0,1]$ (with respect to the 1-dimensional Lebesgue measure)
    \item the partial derivative $\frac{\partial f}{\partial x}$ exists everywhere and is bounded, i.e. there is $G>0$ such that $\abs{\frac{\partial f}{\partial x}}\leq G$.
  \end{enumerate}
  Then for every $x\in\mathbb{R}$ the function $y\rightarrow\frac{\partial f}{\partial x}(x,y)$ is summable on $[0,1]$ and (note that in the following both sides are functions of the variable $x$)
  \[\frac{d}{dx}\left(\int_0^1 f(x,y) \,dy\right)=\int_0^1\frac{\partial f}{\partial x}(x,y) \,dy\]
\end{proposition}
\newpage
\section{Product measures}
We define a measure $(\mu\times\nu)$ on $X\times Y$ by setting, for $A\in\mathcal{A}$ and $S\in\mathcal{S}$,
\[(\mu\times\nu)(A\times S):=\mu(A)\nu(S)\]
\begin{proposition}
  Let $\mathcal{A}$ be a $\sigma$-algebra on $X$ and let $\mathcal{S}$ be a $\sigma$-algebra on $Y$. Then the collection $\mathcal{A}\times\mathcal{S}:=\{A\times S\subset X\times Y\mid A\in\mathcal{A}, S\in\mathcal{S}\}$ is a semi-ring (or semi-algebra, see Chapter 4) on $X\times Y$.
\end{proposition}
\begin{proposition}
  The assignment $(\mu\times nu)$ is a pre-measure on the semi-ring $\mathcal{A}\times\mathcal{B}$.
\end{proposition}
\begin{definition}
  The $\sigma$-algebra generated by the semi-ring $\mathcal{A}\times\mathcal{S}$ is denoted by $\mathcal{A}\otimes\mathcal{S}$ and it is called the product $\sigma$-algebra of $\mathcal{A}$ and $\mathcal{S}$.
\end{definition}
\begin{definition}
  Let $(X,\mathcal{A},\mu)$ and $(Y,\mathcal{S},\nu)$ be $\sigma$-finite measure spaces. The unique triple $(X\times Y,\mathcal{A}\otimes\mathcal{S},\mu\times\nu)$ constructed above is called the product measure space of $(X,\mathcal{A},\mu)$ and $(Y,\mathcal{S},\nu)$.
\end{definition}
\noindent\textbf{Corollary 8.1.1.} The Lebesgue measure on $\mathbb{R}^n$ is the product measure of the Lebesgue measures on $\mathbb{R}^k$ and $\mathbb{R}^d$ whenever $k+d=n$. Moreover $(\mathbb{R}^n,\mathcal{B}(\mathbb{R}^n),\lambda^n)$ is the product measure space of the following two factors:
\[(\mathbb{R}^n,\mathcal{B}(\mathbb{R}^n),\lambda^n)=(\mathbb{R}^k\times\mathbb{R}^d,\mathcal{B}(\mathbb{R}^k)\otimes\mathcal{B}(\mathbb{R}^d),\lambda^k\times\lambda^d)\]
\noindent\textbf{Lemma 8.2.1.} Let $\mathcal{A}\otimes\mathcal{S}$ on $X\times Y$ be the product $\sigma$-algebra defined in the previous section. For $D\subset X\times Y$ we look at 'slices':
\begin{enumerate}
  \item for every $y\in Y$ let $D_y$ denote the set $\{x\in X\mid(x,y)\in D$
  \item for every $x\in X$ let $D_x$ denote the set $\{y\in Y\mid (x,y)\in D\}$
\end{enumerate}
Let $D\in\mathcal{A}\otimes\mathcal{S}$; then for every $y\in Y$ the set $D_y$ is in $\mathcal{A}$ and for every $x\in X$ the set $D_x$ is in $\mathcal{S}$.

\vspace{12pt}

\noindent\textbf{Theorem 8.2.1} (Tonelli's theorem). let $(X\times Y,\mathcal{A}\otimes\mathcal{S},\mu\times\nu)$ be the product measure space of $(X,\mathcal{A},\mu)$ and $(Y,\mathcal{S},\nu)$, where all of these measure spaces are $\sigma$-finite. If $f:X\times Y\rightarrow\overline{\mathbb{R}}$ is $(\mathcal{A}\otimes\mathcal{S})$-measurable and non-negative, then 
\begin{enumerate}
  \item for every $y\in Y$ the function from $X$ to $\overline{\mathbb{R}}$ defined by the assignment $x\rightarrow f(x,y)$ is $\mathcal{A}$-measurable; \\ for every $x\in X$ the function from $Y$ to $\overline{\mathbb{R}}$ defined by the assignment $y\rightarrow f(x,y)$ is $\mathcal{S}$-measurable
  \item the function (from $X$ to $\overline{\mathbb{R}}$) defined by the assignment $x\rightarrow\int_Y f(x,y) \,d\nu(y)$ is $\mathcal{A}$-measurable; \\ the function (from $Y$ to $\overline{\mathbb{R}}$) defined by the assignnment $y\rightarrow\int_X f(x,y) \,d\mu(x)$ is $\mathcal{S}$-measurable
  \item $\int_{X\times Y} f \,d(\mu\times \nu)=\int_X\left(\int_Y f(x,y) \,d\nu(y)\right) \,d\mu(x)=\int_Y\left(\int_X f(x,y) \,d\mu(x)\right) \,d\nu(y)$ where the values are in $[0,+\infty]$
\end{enumerate}
\noindent\textbf{Theorem 8.2.2} (Fubini's theorem). Let $(X\times Y,\mathcal{A}\otimes\mathcal{S},\mu\times\nu)$ be the product measure space of $(X,\mathcal{A},\mu)$ and $(Y,\mathcal{S},\nu)$, where all of these measure spaces are $\sigma$-finite and let $f:X\times Y\rightarrow\overline{\mathbb{R}}$ be $(\mathcal{A}\otimes\mathcal{S})$-measurable. Assume that atleast one of the following integrals is finite:
\[\int_{X\times Y} |f| \,d(\mu\times\nu),
\int_X\left(\int_Y |f|(x,y) \,d\nu(y)\right) \,d\mu(x), \int_Y\left(\int_X |f|(x,y) \,d\mu(x)\right) \,d\nu(y)\],
then $f$ is $(\mu\times\nu)$-summable on $X\times Y$ and all three aboe integrals are finite and coincide. Moreover:
\begin{enumerate}
  \item for $\nu$-almost everywhere $y\in Y$ the function (from $X$ to $\mathbb{R}$) defined by the assignment $x\rightarrow f(x,y)$ is $\mathcal{A}$-measurable and $\mu$-summable; \\
  for $\mu$-almost everywhere $x\in X$ the function (from $Y$ to $\mathbb{R}$) defined by the assignment $y\rightarrow f(x,y)$ is $\mathcal{S}$-measurable and $\nu$-summable
  \item the function defined by the assignment $x\rightarrow \int_Y f(x,y) \,d\nu(y)$ is $\mathcal{A}$-measurable and $\mu$-summable; \\
  the function defined by the assignment $y\rightarrow \int_X f(x,y) \,d\nu(y)$ is $\mathcal{S}$-measurable and $\nu$-summable
  \item $\int_{X\times Y} f \,d(\mu\times\nu) =
  \int_X\left(\int_Y f(x,y) \,d\nu(y)\right) \,d\mu(x) = \int_Y\left(\int_X f(x,y) \,d\mu(x)\right) \,d\nu(y)$,
  where the values are in $(-\infty,+\infty)$
\end{enumerate}

\vspace{12pt}

\noindent\textbf{Theorem 8.2.3} (Fubini's theorem for Lebesgue-measurable functions). Let $f:\mathbb{R}^n\rightarrow\overline{\mathbb{R}}$ be Lebesgue-measurable. Assume that $k+d=n$ ($d,k\in\mathbb{N}*$). Then $f$ is $\lambda^n$-summable if and only if either one of the following integrals is finite:
\[\int_{\mathbb{R}^d}\left(\int_{\mathbb{R}^k}|f|(x,y) \,d\lambda^k(y)\right) \,d\lambda^d(x), \int_{\mathbb{R}^k}\left(\int_{\mathbb{R}^d}|f|(x,y) \,d\lambda^d(x)\right) \,d\lambda^k(y)\]
In that case \\
\small\[\int_{\mathbb{R}^n} f(x,y) \,d\lambda^n=\int_{\mathbb{R}^d}\left(\int_{\mathbb{R}^k}|f|(x,y) \,d\lambda^k(y)\right) \,d\lambda^d(x), \int_{\mathbb{R}^k}\left(\int_{\mathbb{R}^d}|f|(x,y) \,d\lambda^d(x)\right) \,d\lambda^k(y)\]
where all the inner integrals are well-defined and finite almost-everywhere.


\end{document}